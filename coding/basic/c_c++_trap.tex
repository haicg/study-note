%% LyX 2.1.4 created this file.  For more info, see http://www.lyx.org/.
%% Do not edit unless you really know what you are doing.
\documentclass[oneside]{book}
\usepackage[T1]{fontenc}
\setcounter{secnumdepth}{3}
\setcounter{tocdepth}{3}
\usepackage[unicode=true,pdfusetitle,
 bookmarks=true,bookmarksnumbered=true,bookmarksopen=false,
 breaklinks=false,pdfborder={0 0 1},backref=false,colorlinks=false]
 {hyperref}

\makeatletter
%%%%%%%%%%%%%%%%%%%%%%%%%%%%%% Textclass specific LaTeX commands.
\newenvironment{lyxcode}
{\par\begin{list}{}{
\setlength{\rightmargin}{\leftmargin}
\setlength{\listparindent}{0pt}% needed for AMS classes
\raggedright
\setlength{\itemsep}{0pt}
\setlength{\parsep}{0pt}
\normalfont\ttfamily}%
 \item[]}
{\end{list}}

%%%%%%%%%%%%%%%%%%%%%%%%%%%%%% User specified LaTeX commands.
\usepackage{amsmath}
\usepackage{fontspec}
\usepackage{xunicode}
\usepackage{xltxtra}
\usepackage{indentfirst}
\usepackage{xeCJK}
\setmainfont{Times New Roman}
\setCJKmainfont{WenQuanYi Zen Hei}
\setCJKmonofont{WenQuanYi Zen Hei}% 设置缺省中文字体
\setlength{\parindent}{2em}
%\parindent 2em   %段首缩进
\XeTeXlinebreaklocale “zh”
%\XeTeXlinebreakskip = 0pt plus 1pt minus 0.1pt
%\newcommand\Helvetica{\fontspec {Helvetica}}

\usepackage{listings}
\usepackage{color}
\usepackage{xcolor}
\lstset{numbers=left,
numberstyle=\tiny,
keywordstyle=\color{blue!70}, commentstyle=\color{red!50!green!50!blue!50},
frame=single,
rulesepcolor=\color{red!20!green!20!blue!20}
}
\lstset{breaklines}

\makeatother

\begin{document}

\title{C/C++ 不容易发觉的坑}


\author{Lihai}

\maketitle

\chapter{数据类型的隐式转换epackage\{fontspec\} \textbackslash{}usepackage\{xunicode\}
\textbackslash{}usepackage\{xltxtra\} \textbackslash{}usepackage\{indentfirst\}
\textbackslash{}usepackage\{xeCJK\} \textbackslash{}setmainfont\{Times
New Roman\} \textbackslash{}setCJKmainfont\{WenQuanYi Zen Hei\} \textbackslash{}setlength\{\textbackslash{}parindent\}\{2em\}
\%\textbackslash{}parindent 2em \%段首缩进 \textbackslash{}Xe\protect\TeX{}linebreaklocale
“zh” \%\textbackslash{}Xe\protect\TeX{}linebreakskip = 0pt plus 1pt minus
0.1pt \%\textbackslash{}newcommand\textbackslash{}Helvetica\{\textbackslash{}fontspec
\{Helvetica\}\}}

\textbackslash{}usepackage\{listings\} \textbackslash{}usepackage\{color\}
\textbackslash{}usepackage\{xcolor\} \textbackslash{}lstset\{numbers=left,
numberstyle=\textbackslash{}tiny, keywordstyle=\textbackslash{}color\{blue!70\},
commentstyle=\textbackslash{}color\{red!50!green!50!blue!50\}, frame=single,
rulesepcolor=\textbackslash{}color\{red!20!green!20!blue!20\} \} \textbackslash{}lstset\{breaklines\}


\section{枚举类型到底是什么类型}

首先看一个例子:
\begin{lyxcode}
\lstinputlisting[language=c++]{enum_test.cpp}
\end{lyxcode}
是不是觉得这个输出结果应该都相同,都是f0000,但是结果和想象的不一样,执行出来的结果是:
\begin{lyxcode}
A~=~f0000~

B>\textcompwordmark{}>12~=~ffff0000~

C~=~f0000~
\end{lyxcode}
这个结果让人有点想不到明白,为什么高位都会被补1了呢,难道是溢出了,但是算了一下也没有溢出。

那么问题就可能出现在隐式转换上面,这个枚举类型在运算的过程中可能会被转换成什么呢,通过查找ISO C的标准
\begin{quotation}
Usual Arirlmetric Conversions: 

First, if either operand has type long double. the other operand is
convened to long double. 

Otherwise, if either operand has type double, the other operand is
convened to double. 

Otherwise. if either operand has type float, the other operand is
convened to float. 

Otherwise. the integral promotions are performed on both operands.
Then the following rules are applied: If either operand has type unsigned
long int, the other operand is convened to unsigned long int. 

Otherwise. if onr operand has type long int and the other has type
unsigned int. if a long int can represent all values of an unsigned
int. the operand of type unsigned int is converted to long int; if
a long int cannot represent all the values of an unsigned int, both
operands are convened to unsigned long int 

Otherwise. it rithcr operand has type long int, the other operand
is converted to long int 

Otherwisc. if either operand has type unsigned int. the other operand
is convened t o unsigned int 

Otherwise both operands have type int.
\end{quotation}
从上面的内容来说,这边的枚举类型应该是在运算的时候被转换成了int型,但是这边还有一个比较让人疑惑的地方就是C的输出也是经过移位运算的,但是结果没有改变,也就是在这个过程中没有发生隐式转换。可以发现隐式转换的产生是在赋值运算之后,也就是在赋值之前,任何变量都是保持原有的类型。
\begin{thebibliography}{1}
\bibitem[1]{1}ANSI/ISO 9899-1990 American National Stanard for Programming
Languages-C

\bibitem[2]{2}GB/T 15272-94 中华人民共和国国家标准 程序设计语言 C\end{thebibliography}

\end{document}
