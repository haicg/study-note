%% LyX 2.0.8.1 created this file.  For more info, see http://www.lyx.org/.
%% Do not edit unless you really know what you are doing.
\documentclass{article}
\usepackage[T1]{fontenc}
\usepackage[unicode=true,pdfusetitle,
 bookmarks=true,bookmarksnumbered=true,bookmarksopen=false,
 breaklinks=false,pdfborder={0 0 1},backref=false,colorlinks=false]
 {hyperref}

\makeatletter
%%%%%%%%%%%%%%%%%%%%%%%%%%%%%% User specified LaTeX commands.
\usepackage{amsmath}
\usepackage{fontspec}
\usepackage{xunicode}
\usepackage{xltxtra}
\usepackage{indentfirst}
\usepackage{listings}
\usepackage{xeCJK}

\setmainfont{Times New Roman}

\setCJKmainfont{STKaiti}
\setCJKmonofont{STKaiti}% 设置缺省中文字体
\setlength{\parindent}{2em}
%\parindent 2em   %段首缩进

\XeTeXlinebreaklocale “zh”
%\XeTeXlinebreakskip = 0pt plus 1pt minus 0.1pt
%\newcommand\Helvetica{\fontspec {Helvetica}}

\makeatother

\usepackage{xunicode}
\begin{document}

\title{Zwave SDK 学习记录}

\maketitle

\part{基础知识}


\section{节点类型}

从协议的角度区分,总共有7种节点类型,分别是Portable Controller nodes, Static Controller
nodes, Bridge Controller nodes, Routing Slave nodes, and Enhanced
232 Slave nodes. (七种?)。关于congtroller的设备框架图见文献1中的Page13:
\begin{enumerate}
\item Portable Controller nodes 软件角度来看,主要有两个部分组成,控制应用层(Controller application
)+控制的底层支持(zwave协议层和一些控制数据)。


从硬件的角度来看需要一个外扩NVM。

\item Static Controller nodes 软件角度来看,主要有两个部分组成,静态控制器应用层(Static Controller
application)+控制的底层支持(zwave协议层和一些控制数据)。


硬件的角度来看需要一个外扩NVM。 和上面 Portable Controller nodes 节点之间的区别就是这种类型的节点必须是用电源供电的设备,
不能使用电池供电。 这个特性使得他可以 在Zwave网络担当Primary Conroller (在网络中可以为Slave nodes
指定路由),同时还可以担当网络中的SUC(在网络中可以更新Seconary controller 的信息,不过在新版的SIS 代替了这个SUC)

\item Z-Wave Bridge Controller Node 这个是基于Static Controller node 功能的一个扩展类型,扩展了桥接多种类型的网络的功能,例如将Zwave网络和UPNP
的网络进行桥接。
\item Routing Slave nodes 一般用于功能简单设备,在设备中没有保存整个网络的路由表。其中数据发送到目的地主要依靠和controller的一个association,并在添加时候由控制器给定一个路由设置。
硬件上没有extern NVM ,只有一个64bytes的MTP。初始的时候,节点内的homeId是一个随机值,而node Id 必须为0.在添加某个zwave网络的时候,其中的Node
Id和 Home Id 都是由主控制器来设置。用于电池设备时,会进入睡眠模式,同时具有WUT功能。
\item Z-Wave Enhanced 232 Slave Node 是基于Routing Slave nodes, 主要区别就是(return
route assignment of up to 232 destination nodes instead of 5)返回的路由,从最多5个节点,到最多232个节点,也就是路由的路线变长了。从硬件来说,比Routing
Slave nodes多了一个extern nvm。
\end{enumerate}

\section{添加和删除节点的流程及网络自动更新}


\subsection{添加流程}


\subsection{删除流程}


\subsection{网络自动更新}

对于网络中只有一个controller的时候,基本没有什么问题。但是对于网络中存在多个controler,这个时候就需要网络有一个SIS
,这个设备要负责整个网络的自动更新,首先添加节点设备的controller需要向SIS 主动上报相关信息,然后其余的controller可以向SIS
去询问是否有更新, 这边可以看出, 这边其他控制器的更 新不一定 是同步的,可能存在某些延时(参见文献1的3.7.7)。


\section{CenterController with serial api host}


\subsection{Frame between the host and Zwave chip}

主要有下面四种帧数据类型:ACK,NAK,CAN 和数据帧\cite{2}
\begin{enumerate}
\item ACK表示接收方已经接收一条有效的数据帧,需要发送方再次发送
\item NAK表示接收放接收的是一条错误的数据帧,需要发送方再次发送
\item CAN表示接收端主动丢弃有效数据帧,需要发送端至少等待一个周期之后再发送。例如:Zwave节点正在等待一个ACK,但是却收到一个有效数据帧,此时Zwave节点会给主机回一条CAN
帧。主机收到此消息之后需要等待一个周期之后 再进行发送。
\item 数据帧包含主机要发送的命令和相应的参数,由SOF(0x01)+length+type+serial api Command Id+Serial
API Command Parameter 1 ...N+Checksum\end{enumerate}
\begin{thebibliography}{1}
\bibitem[1]{1}SD, INS12308-6, Instruction, Z-Wave 500 Series Appl.
Prg. Guide v6.51.03 :12-13

\bibitem[2]{2}SD,INS12350 -6,Serial API Host Appl. Prg. Guide :12-13\end{thebibliography}

\end{document}
