%% LyX 2.0.8.1 created this file.  For more info, see http://www.lyx.org/.
%% Do not edit unless you really know what you are doing.
\documentclass{article}
\usepackage[T1]{fontenc}
\usepackage{graphicx}
\usepackage[unicode=true,pdfusetitle,
 bookmarks=true,bookmarksnumbered=true,bookmarksopen=false,
 breaklinks=false,pdfborder={0 0 1},backref=false,colorlinks=false]
 {hyperref}

\makeatletter
%%%%%%%%%%%%%%%%%%%%%%%%%%%%%% Textclass specific LaTeX commands.
\newenvironment{lyxcode}
{\par\begin{list}{}{
\setlength{\rightmargin}{\leftmargin}
\setlength{\listparindent}{0pt}% needed for AMS classes
\raggedright
\setlength{\itemsep}{0pt}
\setlength{\parsep}{0pt}
\normalfont\ttfamily}%
 \item[]}
{\end{list}}

%%%%%%%%%%%%%%%%%%%%%%%%%%%%%% User specified LaTeX commands.
\usepackage{amsmath}
\usepackage{fontspec}
\usepackage{xunicode}
\usepackage{xltxtra}
\usepackage{indentfirst}
\usepackage{listings}
\usepackage{xeCJK}

\setmainfont{Times New Roman}

\setCJKmainfont{STKaiti}
\setCJKmonofont{STKaiti}% 设置缺省中文字体
\parindent 2em   %段首缩进

\XeTeXlinebreaklocale “zh”
%\XeTeXlinebreakskip = 0pt plus 1pt minus 0.1pt
\newcommand\Helvetica{\fontspec {Helvetica}}

\makeatother

\usepackage{xunicode}
\begin{document}

\title{LATEX 及LYX 使用问题总结}

\maketitle

\section{运行环境:}


\subparagraph{系统:Ubuntu14.04 x86\_64}


\section{LYX的中文支持问题}

最后采用的是LYX和Xetex的组合,这种方法可以简单有效的解决中文乱码问题\cite{1}。

这边一定要有个需要注意的问题,就是先安装Xetex,然后再安装LYX,因为LYX在安装完成后,第一次使用的时候会去扫描当前的系统环境。

LYX肯定有一个初始化的文件夹,估计将那个删除也会使LYX 重新进行一次系统环境扫描(个人推测)。

安装Xetex的命令即安装LYX的命令是\cite{2}:
\begin{lyxcode}
sudo~apt-get~install~texlive-xetex~(Install~xetex)

sudo~add-apt-repository~ppa:lyx-devel/release~~(Add~PPA~for~the~'Lyx~release')

sudo~apt-get~update

sudo~apt-get~install~lyx2.0~(v1~or~v2.0~on~you,~install~from~ppa)
\end{lyxcode}
到这一步已经有了一个基本的写作环境,但是对于中文文档,还需要在LYX的文档导言区进行一些设置:
\begin{lyxcode}
\textbackslash{}usepackage\{amsmath\}~

\textbackslash{}usepackage\{fontspec\}~

\textbackslash{}usepackage\{xunicode\}~

\textbackslash{}usepackage\{xltxtra\}~

\textbackslash{}usepackage\{indentfirst\}~\%第一段也可以缩进

\textbackslash{}usepackage\{listings\}~

\textbackslash{}usepackage\{xeCJK\}

\textbackslash{}setmainfont\{Times~New~Roman\}

\textbackslash{}setCJKmainfont\{STKaiti\}~

\textbackslash{}setCJKmonofont\{STKaiti\}~\%~设置缺省中文字体~

\textbackslash{}parindent~2em~~~\%段首缩进

\textbackslash{}Xe\TeX{}linebreaklocale~“zh”~\textbackslash{}Xe\TeX{}linebreakskip~=~0pt~plus~1pt~minus~0.1pt~

\textbackslash{}newcommand\textbackslash{}Helvetica\{\textbackslash{}fontspec~\{Helvetica\}\}
\end{lyxcode}

\paragraph{上面的这段内容主要是设置中文的默认字体是华文楷体,英文的默认字体是Times New Roman,段首缩进2个字符。}


\section{LYX 的中文字体问题}

因为Ubuntu中默认的中文字体很少,需要自己手动进行安装。

具体的安装步骤:
\begin{lyxcode}
sudo~apt-get~install~xfonts-wqy~ttf-wqy-microhei~ttf-wqy-zenhei

fc-cache

sudo~fc-cache~-f~-s~-v

fc-list~:lang=zh
\end{lyxcode}
上面是通过第一条命令去下载免费的字体库的,也可以自己从Windows 系统下面将Fonts文件夹下的字体拷贝到/usr/share/fonts/win/目录下,然后继续执行下面的指令。


\section{LYX及Latex的入门教程}
\begin{enumerate}
\item \href{http://www.ibm.com/developerworks/cn/opensource/os-lyx/index.html}{使用 LyX 以图形形式创建 LaTeX 文档}
\item \href{http://blog.sina.com.cn/s/blog_61736b5d0100lsm7.html}{XETEX / LaTEX 中文排版之胡言乱语}
\item \href{http://blog.sina.com.cn/s/blog_630e5dec0100w3jl.html}{LyX 入门教程 (翻译 The LyX Tutorial)}
\item \href{http://www.loyhome.com/\%E6\%9B\%B4\%E6\%94\%B9lyx\%E7\%9A\%84\%E5\%BF\%AB\%E6\%8D\%B7\%E5\%B7\%A5\%E5\%85\%B7\%E6\%A0\%8F\%EF\%BC\%88\%E5\%8A\%A0\%E5\%85\%A5\%E5\%AF\%BC\%E5\%87\%BA\%E6\%8C\%89\%E9\%92\%AE\%EF\%BC\%89-2/}{更改LyX的快捷工具栏(加入“导出”按钮)}
\item \href{https://en.wikibooks.org/wiki/LaTeX/Source_Code_Listings}{LaTeX/Source Code Listings}主要介绍如何在latex文档中插入源代码。
\item \href{http://blog.csdn.net/xiazdong/article/details/8892105}{【LaTeX入门】05、换行、换段、换页、首行缩进等命令} 
\end{enumerate}

\section{LYX遇到的一些特别的问题}
\begin{enumerate}
\item 首行缩进的问题


在段落缩进方面输出PDF 和输出HTML得到的样式不相同,这个折腾了我好久,还没有解决如何在输出到HTML的文件中,完成自动的段落首行缩进的问题。后来查询了相关资料,发现得到HTML的工具不是Xelatex,这样出现这个问题也就能解释了。


解决的方法就是在开始的地方空一行(插入一个格式化符号就可以了),如果不需要考虑HTML 输出的话,完全可以忽视这个问题。

\item 生成的PDF 文档中的书签没有编号


需要自己配置一下,在Doucument ->setting->PDF Properties

\end{enumerate}
\includegraphics[scale=0.5]{\string"images/Screenshot from 2016-03-15 11:34:37\string".png}
\begin{thebibliography}{1}
\bibitem[1]{1}\href{http://linux-wiki.cn/wiki/LaTeX\%E4\%B8\%AD\%E6\%96\%87\%E6\%8E\%92\%E7\%89\%88\%EF\%BC\%88\%E4\%BD\%BF\%E7\%94\%A8XeTeX\%EF\%BC\%89}{LaTeX中文排版(使用XeTeX)}

\bibitem[2]{2}\href{http://wiki.lyx.org/LyX/LyXOnUbuntu\#toc3}{LyXOnUbuntu 官方帮助页面}

\bibitem[3]{3}\href{http://www.cnblogs.com/biaoyu/archive/2012/04/28/2475318.html}{Alex-LyX中文问题 }\end{thebibliography}

\end{document}
